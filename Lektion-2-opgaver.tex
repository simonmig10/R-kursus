% Options for packages loaded elsewhere
\PassOptionsToPackage{unicode}{hyperref}
\PassOptionsToPackage{hyphens}{url}
\PassOptionsToPackage{dvipsnames,svgnames*,x11names*}{xcolor}
%
\documentclass[
  12pt,
]{article}
\usepackage{amsmath,amssymb}
\usepackage{lmodern}
\usepackage{ifxetex,ifluatex}
\ifnum 0\ifxetex 1\fi\ifluatex 1\fi=0 % if pdftex
  \usepackage[T1]{fontenc}
  \usepackage[utf8]{inputenc}
  \usepackage{textcomp} % provide euro and other symbols
\else % if luatex or xetex
  \usepackage{unicode-math}
  \defaultfontfeatures{Scale=MatchLowercase}
  \defaultfontfeatures[\rmfamily]{Ligatures=TeX,Scale=1}
\fi
% Use upquote if available, for straight quotes in verbatim environments
\IfFileExists{upquote.sty}{\usepackage{upquote}}{}
\IfFileExists{microtype.sty}{% use microtype if available
  \usepackage[]{microtype}
  \UseMicrotypeSet[protrusion]{basicmath} % disable protrusion for tt fonts
}{}
\makeatletter
\@ifundefined{KOMAClassName}{% if non-KOMA class
  \IfFileExists{parskip.sty}{%
    \usepackage{parskip}
  }{% else
    \setlength{\parindent}{0pt}
    \setlength{\parskip}{6pt plus 2pt minus 1pt}}
}{% if KOMA class
  \KOMAoptions{parskip=half}}
\makeatother
\usepackage{xcolor}
\IfFileExists{xurl.sty}{\usepackage{xurl}}{} % add URL line breaks if available
\IfFileExists{bookmark.sty}{\usepackage{bookmark}}{\usepackage{hyperref}}
\hypersetup{
  pdftitle={Lektion 2 opgaver},
  colorlinks=true,
  linkcolor=Maroon,
  filecolor=Maroon,
  citecolor=blue,
  urlcolor=blue,
  pdfcreator={LaTeX via pandoc}}
\urlstyle{same} % disable monospaced font for URLs
\usepackage[margin=1in]{geometry}
\usepackage{color}
\usepackage{fancyvrb}
\newcommand{\VerbBar}{|}
\newcommand{\VERB}{\Verb[commandchars=\\\{\}]}
\DefineVerbatimEnvironment{Highlighting}{Verbatim}{commandchars=\\\{\}}
% Add ',fontsize=\small' for more characters per line
\usepackage{framed}
\definecolor{shadecolor}{RGB}{248,248,248}
\newenvironment{Shaded}{\begin{snugshade}}{\end{snugshade}}
\newcommand{\AlertTok}[1]{\textcolor[rgb]{0.94,0.16,0.16}{#1}}
\newcommand{\AnnotationTok}[1]{\textcolor[rgb]{0.56,0.35,0.01}{\textbf{\textit{#1}}}}
\newcommand{\AttributeTok}[1]{\textcolor[rgb]{0.77,0.63,0.00}{#1}}
\newcommand{\BaseNTok}[1]{\textcolor[rgb]{0.00,0.00,0.81}{#1}}
\newcommand{\BuiltInTok}[1]{#1}
\newcommand{\CharTok}[1]{\textcolor[rgb]{0.31,0.60,0.02}{#1}}
\newcommand{\CommentTok}[1]{\textcolor[rgb]{0.56,0.35,0.01}{\textit{#1}}}
\newcommand{\CommentVarTok}[1]{\textcolor[rgb]{0.56,0.35,0.01}{\textbf{\textit{#1}}}}
\newcommand{\ConstantTok}[1]{\textcolor[rgb]{0.00,0.00,0.00}{#1}}
\newcommand{\ControlFlowTok}[1]{\textcolor[rgb]{0.13,0.29,0.53}{\textbf{#1}}}
\newcommand{\DataTypeTok}[1]{\textcolor[rgb]{0.13,0.29,0.53}{#1}}
\newcommand{\DecValTok}[1]{\textcolor[rgb]{0.00,0.00,0.81}{#1}}
\newcommand{\DocumentationTok}[1]{\textcolor[rgb]{0.56,0.35,0.01}{\textbf{\textit{#1}}}}
\newcommand{\ErrorTok}[1]{\textcolor[rgb]{0.64,0.00,0.00}{\textbf{#1}}}
\newcommand{\ExtensionTok}[1]{#1}
\newcommand{\FloatTok}[1]{\textcolor[rgb]{0.00,0.00,0.81}{#1}}
\newcommand{\FunctionTok}[1]{\textcolor[rgb]{0.00,0.00,0.00}{#1}}
\newcommand{\ImportTok}[1]{#1}
\newcommand{\InformationTok}[1]{\textcolor[rgb]{0.56,0.35,0.01}{\textbf{\textit{#1}}}}
\newcommand{\KeywordTok}[1]{\textcolor[rgb]{0.13,0.29,0.53}{\textbf{#1}}}
\newcommand{\NormalTok}[1]{#1}
\newcommand{\OperatorTok}[1]{\textcolor[rgb]{0.81,0.36,0.00}{\textbf{#1}}}
\newcommand{\OtherTok}[1]{\textcolor[rgb]{0.56,0.35,0.01}{#1}}
\newcommand{\PreprocessorTok}[1]{\textcolor[rgb]{0.56,0.35,0.01}{\textit{#1}}}
\newcommand{\RegionMarkerTok}[1]{#1}
\newcommand{\SpecialCharTok}[1]{\textcolor[rgb]{0.00,0.00,0.00}{#1}}
\newcommand{\SpecialStringTok}[1]{\textcolor[rgb]{0.31,0.60,0.02}{#1}}
\newcommand{\StringTok}[1]{\textcolor[rgb]{0.31,0.60,0.02}{#1}}
\newcommand{\VariableTok}[1]{\textcolor[rgb]{0.00,0.00,0.00}{#1}}
\newcommand{\VerbatimStringTok}[1]{\textcolor[rgb]{0.31,0.60,0.02}{#1}}
\newcommand{\WarningTok}[1]{\textcolor[rgb]{0.56,0.35,0.01}{\textbf{\textit{#1}}}}
\usepackage{graphicx}
\makeatletter
\def\maxwidth{\ifdim\Gin@nat@width>\linewidth\linewidth\else\Gin@nat@width\fi}
\def\maxheight{\ifdim\Gin@nat@height>\textheight\textheight\else\Gin@nat@height\fi}
\makeatother
% Scale images if necessary, so that they will not overflow the page
% margins by default, and it is still possible to overwrite the defaults
% using explicit options in \includegraphics[width, height, ...]{}
\setkeys{Gin}{width=\maxwidth,height=\maxheight,keepaspectratio}
% Set default figure placement to htbp
\makeatletter
\def\fps@figure{htbp}
\makeatother
\setlength{\emergencystretch}{3em} % prevent overfull lines
\providecommand{\tightlist}{%
  \setlength{\itemsep}{0pt}\setlength{\parskip}{0pt}}
\setcounter{secnumdepth}{5}
\usepackage{fancyhdr}
\pagestyle{fancy}
\usepackage{color}
\usepackage{dcolumn}
\usepackage{here}
\usepackage{longtable}
\usepackage{subfig}
\usepackage{caption}
\captionsetup{skip=2pt,labelsep=space,justification=justified,singlelinecheck=off}
\captionsetup[subfigure]{labelformat=empty}
\captionsetup[figure]{labelformat=empty}
\DeclareCaptionFont{scriptsize}{\scriptsize}
\captionsetup[subfigure]{font=scriptsize}
\ifluatex
  \usepackage{selnolig}  % disable illegal ligatures
\fi
\usepackage[]{natbib}
\bibliographystyle{apalike}

\title{Lektion 2 opgaver}
\author{\emph{Simon Fløj Thomsen}\footnote{Aalborg University,
  \href{mailto:sft@business.aau.dk}{\nolinkurl{sft@business.aau.dk}},
  MaMTEP}}
\date{\emph{oktober 26, 2022}}

\begin{document}
\maketitle
\begin{abstract}
\begingroup Formålet med dette dokument er at give en introduktion til
anvendelsen af R-markdown til fremtidige projekter \endgroup
\end{abstract}

\hypertarget{opgave-1-klarguxf8ring-af-r-markdown}{%
\section{Opgave 1 Klargøring af
R-markdown}\label{opgave-1-klarguxf8ring-af-r-markdown}}

Sørg for du er i dit projekt vi lavede sidste gang!

\hypertarget{a-pakker}{%
\subsection{A) pakker}\label{a-pakker}}

\begin{enumerate}
\def\labelenumi{\arabic{enumi}.}
\tightlist
\item
  Åben et nyt r-markdown og give navnet ``Opgaver 2 R-kursus''
\item
  Slet koden givet til at starte med, men behold Yammel koden.
\item
  Lav overskrift ``Packages'' og load følgende pakker:
\end{enumerate}

\begin{Shaded}
\begin{Highlighting}[]
\FunctionTok{library}\NormalTok{(car)}
\FunctionTok{library}\NormalTok{(readxl)}
\FunctionTok{library}\NormalTok{(tseries)}
\end{Highlighting}
\end{Shaded}

\begin{enumerate}
\def\labelenumi{\arabic{enumi}.}
\setcounter{enumi}{3}
\tightlist
\item
  Skriv ``message = False'' i r-chunk, for at undgå ligegyldigt output
  af kode.
\item
  Skriv ``echo = False'' i r-chunk, for at undgå ligegyldig kode.
\end{enumerate}

\hypertarget{b-data}{%
\subsection{B) Data}\label{b-data}}

\begin{enumerate}
\def\labelenumi{\arabic{enumi}.}
\tightlist
\item
  Importer data for BNP, Export og Import med følgende kode: (Jeg bruger
  data fra lektion 1 vi havde)
\end{enumerate}

\begin{Shaded}
\begin{Highlighting}[]
\NormalTok{BNP }\OtherTok{\textless{}{-}} \FunctionTok{read\_excel}\NormalTok{(}\StringTok{"bnp.xlsx"}\NormalTok{)}
\NormalTok{X }\OtherTok{\textless{}{-}} \FunctionTok{read\_excel}\NormalTok{(}\StringTok{"Export.xlsx"}\NormalTok{)}
\NormalTok{IM }\OtherTok{\textless{}{-}} \FunctionTok{read\_excel}\NormalTok{(}\StringTok{"Import.xlsx"}\NormalTok{)}
\end{Highlighting}
\end{Shaded}

\begin{enumerate}
\def\labelenumi{\arabic{enumi}.}
\setcounter{enumi}{1}
\tightlist
\item
  definer de 3 tidsserier ved brug af ts() funktionen
\end{enumerate}

\begin{Shaded}
\begin{Highlighting}[]
\NormalTok{bnp}\OtherTok{=} \FunctionTok{ts}\NormalTok{(BNP}\SpecialCharTok{$}\NormalTok{BNP, }\AttributeTok{start =} \DecValTok{1966}\NormalTok{, }\AttributeTok{frequency =} \DecValTok{1}\NormalTok{ )}
\NormalTok{x}\OtherTok{=} \FunctionTok{ts}\NormalTok{(X}\SpecialCharTok{$}\NormalTok{X, }\AttributeTok{start =} \DecValTok{1966}\NormalTok{, }\AttributeTok{frequency =} \DecValTok{1}\NormalTok{ )}
\NormalTok{im}\OtherTok{=} \FunctionTok{ts}\NormalTok{(IM}\SpecialCharTok{$}\NormalTok{Import, }\AttributeTok{start =} \DecValTok{1966}\NormalTok{, }\AttributeTok{frequency =} \DecValTok{1}\NormalTok{ )}
\end{Highlighting}
\end{Shaded}

Normalt hvis der er brug for data manipulation som vi kigger på næste
gang gør jeg det her!

\hypertarget{visualisering-af-data}{%
\section{Visualisering af Data}\label{visualisering-af-data}}

\begin{enumerate}
\def\labelenumi{\arabic{enumi}.}
\item
  Lav 4 plots du syntes giver mening, da vi nu har tidsserier behøves i
  ik definere ``YEAR'' som sidste gang. (Husk ved brug af lines() skal
  det være på samme linje!)
\item
  Du kan skifte størrelsen på dine plots ved brug af ``fig.width=5'' og
  ``fig.height=5'' (Dette kan også sættes som en generel setting i
  Yammel koden)
\item
  Find et billede på nettet/din computer du vil bruge i dit dokument,
  gem det i din \textbf{Projekt} mappe ved navnet ``billede.jpg''.
\item
  Bru nedenstående kode for at sætte billedet ind i dokumentet. (Hint:
  dpi= 300 bestemmer størrlesen af billedet, jo mindre tal jo større
  billede)
\end{enumerate}

\begin{Shaded}
\begin{Highlighting}[]
\NormalTok{knitr}\SpecialCharTok{::}\FunctionTok{include\_graphics}\NormalTok{(}\FunctionTok{rep}\NormalTok{(}\StringTok{"billede.jpg"}\NormalTok{, }\DecValTok{1}\NormalTok{), }\AttributeTok{dpi =} \DecValTok{300}\NormalTok{)}
\end{Highlighting}
\end{Shaded}

\hypertarget{analyse}{%
\section{Analyse}\label{analyse}}

\begin{enumerate}
\def\labelenumi{\arabic{enumi}.}
\tightlist
\item
  Udregn correlation mellem de 3 variable:
\end{enumerate}

\begin{Shaded}
\begin{Highlighting}[]
\FunctionTok{cor}\NormalTok{(bnp,x)}
\end{Highlighting}
\end{Shaded}

\begin{verbatim}
## [1] 0.9791313
\end{verbatim}

\begin{Shaded}
\begin{Highlighting}[]
\FunctionTok{cor}\NormalTok{(bnp,im)}
\end{Highlighting}
\end{Shaded}

\begin{verbatim}
## [1] 0.9619662
\end{verbatim}

\begin{Shaded}
\begin{Highlighting}[]
\FunctionTok{cor}\NormalTok{(x,im)}
\end{Highlighting}
\end{Shaded}

\begin{verbatim}
## [1] 0.9955177
\end{verbatim}

\begin{enumerate}
\def\labelenumi{\arabic{enumi}.}
\setcounter{enumi}{1}
\tightlist
\item
  presenter i en tabel ved brug af nedenstående kode (Dette skal ikke
  være i R-kode!)
\end{enumerate}

  \bibliography{Bibliography.bib}

\end{document}
